\documentclass[10pt,fleqn]{article} % Default font size and left-justified equations
\usepackage[%
    pdftitle={Energétique},
    pdfauthor={Xavier Pessoles}]{hyperref}

    
\input{style/new_style}
\input{style/macros_SII}
\usepackage{multicol}
\usepackage{siunitx}
%\usepackage{picins}
\fichetrue
%\fichefalse

\proftrue

\proffalse

\tdtrue
%\tdfalse

\courstrue
\coursfalse


\def\classe{\textsf{PTSI}}
\def\xxnumpartie{}%Cycle --}
\def\xxpartie{ }

\def\xxnumchapitre{}%Chapitre -- \vspace{.2cm}}
\def\xxchapitre{\hspace{.12cm} }

\def\discipline{Sciences \\Industrielles de \\ l'Ingénieur}
\def\xxtete{Sciences Industrielles de l'Ingénieur}


  
\def\xxposongletx{2}
\def\xxposonglettext{1.45}
\def\xxposonglety{20}
%\def\xxonglet{Part. 1 -- Ch. 3}
\def\xxonglet{\textsf{}}%Cycle 05}}

\def\xxactivite{Colle 02}
\def\xxauteur{\textsl{Xavier Pessoles}}


\def\xxtitreexo{}%Siège motorisé}
\def\xxsourceexo{\hspace{.2cm} \footnotesize{}}%BTS CPI 2018}}


\def\xxcompetences{%
\vspace{-.5cm}
\footnotesize{
\textsl{%
\textbf{Savoirs et compétences :}\\
\vspace{-.2cm}
%\begin{itemize}[label=\ding{112},font=\color{ocre}] 
%\item Mod2.C18.SF1 : Déterminer l’énergie cinétique d’un solide, ou d’un ensemble de solides, dans son mouvement par rapport à un autre solide.
%\item Res1.C1.SF1 : Proposer une démarche permettant la détermination de la loi de mouvement.
%\item Mod1.C5.SF2 : Déterminer la puissance des actions mécaniques extérieures à un solide ou à un ensemble de solides, dans son mouvement rapport à un autre solide.
%\item Mod1.C5.SF3 : Déterminer la puissance des actions mécaniques intérieures à un ensemble de solides.
%\end{itemize}
}}}

\def\xxfigures{
\includegraphics[width=.5\textwidth]{images/fig_01}
}%figues de la page de garde


\def\xxpied{%
%Cycle 05 -- Modélisation mécanique -- Énergétique\\% afin de valider leurs performances.\\
%Chapitre 1 -- \xxactivite%
}

\setcounter{secnumdepth}{5}
%---------------------------------------------------------------------------


\begin{document}
%\chapterimage{png/Fond_Cin}
\input{style/new_pagegarde}
\vspace{5.5cm}
\pagestyle{fancy}
\thispagestyle{plain}


\def\columnseprulecolor{\color{ocre}}
\setlength{\columnseprule}{0.4pt} 

%\ifprof
%\else
\begin{multicols}{2}
%\fi

\section*{Train simple}
\setcounter{exo}{0}
Soit le train d'engrenages suivant. 
\begin{center}
\includegraphics[width=.7\linewidth]{images/TrainSimple_02}
\end{center}


\subparagraph{}
\textit{Déterminer $\dfrac{\omega_{4/0}}{\omega_{1/0}}$ en fonction du nombre de dents des roues dentées.}
\ifprof
\begin{corrige}
On a $\dfrac{\omega_{4/0}}{\omega_{1/0}}=-\dfrac{Z_1Z_{22}}{Z_4Z_{21}}$.
\end{corrige}
\else
\fi

\subparagraph{}
\textit{Donner une relation géométrique entre $Z_1$, $Z_{21}$, $Z_{22}$ et $Z_4$ permettant de garantir le fonctionnement du train d'engrenages. }
\ifprof
\begin{corrige}
On a $Z_1+Z_{21}+Z_{22}= Z_4$.
\end{corrige}
\else
\fi




\section*{ Calcul de moments}
\setcounter{exo}{0}
\setcounter{subparagraph}{0}
On donne la structure suivante : 
\begin{center}
\includegraphics[width=.8\linewidth]{images/fig_23}
\end{center}

\subparagraph{}
\textit{Déterminer $\vect{\mathcal{M}\left(A,\vect{F} \right)}$.}



On donne la structure suivante : 
\begin{center}
\includegraphics[width=.8\linewidth]{images/fig_24}
\end{center}


\subparagraph{}
\textit{Déterminer $\vect{\mathcal{M}\left(B,\vect{F} \right)}$.}




\section*{Solide en rotation}
\setcounter{subparagraph}{0}
\setcounter{exo}{0}

 Une machine est entraînée par un moteur électrique de fréquence nominale $1\,500\; \text{tr}\,\text{min}^{-1}$. Celui-ci exerce au démarrage un couple moteur constant de $40\; \text{N}\,\,\text{m}$. Le moment d’inertie de l’ensemble de la chaîne cinématique rapporté à l’axe du rotor est de $12,5\; \text{kg}\,\text{m}^2$. Le couple résistant dû aux frottements est supposé constant et égal à $4\, \text{N} \, \text{m}$.

\subparagraph{}
\textit{Calculer l’accélération du moteur pendant le démarrage.}

\subparagraph{}
\textit{Calculer le temps mis pour atteindre la fréquence nominale.}



\section*{Lois de Kirchoff}
\setcounter{exo}{0}

\subparagraph*{}
\textit{Sur le circuit suivant, déterminer les courants dans chacune des branches et la tension aux bornes de tous les dipôles en fonction de $E$ et des différentes résistances $R_i$.}
\begin{center}
\includegraphics[width=\linewidth]{images/fig_02}
\end{center}


\section*{Résistance équivalente}
\setcounter{exo}{0}
\textit{Déterminer la résistance équivalente du montage suivant.}
\begin{center}
\includegraphics[width=\linewidth]{images/fig_06}
\end{center}


\end{multicols}

\end{document}

\subparagraph{}\textit{}
\ifprof
\begin{corrige}~\\
\end{corrige}
\else
\fi




\subparagraph{}\textit{}
\ifprof
\begin{corrige}~\\
\end{corrige}
\else
\fi

\subparagraph{}\textit{}
\ifprof
\begin{corrige}~\\
\end{corrige}
\else
\fi

\subparagraph{}\textit{}
\ifprof
\begin{corrige}~\\
\end{corrige}
\else
\fi

\subparagraph{}\textit{}
\ifprof
\begin{corrige}~\\
\end{corrige}
\else
\fi

\subparagraph{}\textit{}
\ifprof
\begin{corrige}~\\
\end{corrige}
\else
\fi

\subparagraph{}\textit{}
\ifprof
\begin{corrige}~\\
\end{corrige}
\else
\fi
\begin{center}
%\includegraphics[width=\linewidth]{images/fig_05}
\end{center}
